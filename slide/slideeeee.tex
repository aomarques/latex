\documentclass[12pt]{beamer}
\usetheme{PaloAlto}
\usepackage[utf8]{inputenc}
\usepackage[portuguese]{babel}
\usepackage[T1]{fontenc}
\usepackage{amsmath}
\usepackage{amsfonts}
\usepackage{amssymb}
\usepackage{graphicx}
\author{Álan Marques}
\title{Exemplo de Slide em LaTex}
%\setbeamercovered{transparent} 
%\setbeamertemplate{navigation symbols}{} 
%\logo{} 
%\institute{} 
%\date{} 
%\subject{}
\subtitle{Slides}
\institute{Universidade Estadual do Norte do Paraná}


\begin{document}


\begin{frame}
	\titlepage
\end{frame}

\begin{frame}{Introdução}
	\begin{itemize}
\item O que é Beamer?
\item Quais suas vantagens?
	
	\end{itemize}
\end{frame}




%\begin{frame}
%\tableofcontents
%\end{frame}

\begin{frame}
	\begin{block}{Fórmula}
	\begin{equation}
		a.b = ab
	\end{equation}
	\end{block}
\end{frame}

\begin{frame}{Integrais de Riemann}
\begin{itemize}
	\item Texto1
	\item Texto2
\end{itemize}

	\begin{block}{Fórmula}
	\begin{equation}
		\int^b_a f(x).g'(dx)dx = [f(x)g(x)]^b_a - \int^b_a f'(x).g(x)dx
	\end{equation}
	\end{block}
\begin{figure}[h]
	\begin{flushright}
		\includegraphics[width=1cm]{heman.png}
	\end{flushright}
	\end{figure}
	
\end{frame}


\end{document}